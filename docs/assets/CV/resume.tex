%%%%%%%%%%%%%%%%%%%%%%%%%%%%%%%%%%%%%%%%%%%%%%%%%%%%%%%%%%%%%%%%%%%%%%%%
%%%%%%%%%%%%%%%%%%%%%% Simple LaTeX CV Template %%%%%%%%%%%%%%%%%%%%%%%%
%%%%%%%%%%%%%%%%%%%%%%%%%%%%%%%%%%%%%%%%%%%%%%%%%%%%%%%%%%%%%%%%%%%%%%%%

%%%%%%%%%%%%%%%%%%%%%%%%%%%%%%%%%%%%%%%%%%%%%%%%%%%%%%%%%%%%%%%%%%%%%%%%
%% NOTE: If you find that it says                                     %%
%%                                                                    %%
%%                           1 of ??                                  %%
%%                                                                    %%
%% at the bottom of your first page, this means that the AUX file     %%
%% was not available when you ran LaTeX on this source. Simply RERUN  %%
%% LaTeX to get the ``??'' replaced with the number of the last page  %%
%% of the document. The AUX file will be generated on the first run   %%
%% of LaTeX and used on the second run to fill in all of the          %%
%% references.                                                        %%
%%%%%%%%%%%%%%%%%%%%%%%%%%%%%%%%%%%%%%%%%%%%%%%%%%%%%%%%%%%%%%%%%%%%%%%%

%%%%%%%%%%%%%%%%%%%%%%%%%%%% Document Setup %%%%%%%%%%%%%%%%%%%%%%%%%%%%

% Don't like 10pt? Try 11pt or 12pt
\documentclass[10pt]{article}

% This is a helpful package that puts math inside length specifications
\usepackage{calc}
\usepackage{pifont}
\usepackage{marvosym}


% Simpler bibsection for CV sections
% (thanks to natbib for inspiration)
\makeatletter
\newlength{\bibhang}
\setlength{\bibhang}{1em}
\newlength{\bibsep}
 {\@listi \global\bibsep\itemsep \global\advance\bibsep by\parsep}
\newenvironment{bibsection}%
        {\vspace{-\baselineskip}\begin{list}{}{%
       \setlength{\leftmargin}{\bibhang}%
       \setlength{\itemindent}{-\leftmargin}%
       \setlength{\itemsep}{\bibsep}%
       \setlength{\parsep}{\z@}%
        \setlength{\partopsep}{0pt}%
        \setlength{\topsep}{0pt}}}
        {\end{list}\vspace{-.6\baselineskip}}
\makeatother

% Layout: Puts the section titles on left side of page
\reversemarginpar

%
%         PAPER SIZE, PAGE NUMBER, AND DOCUMENT LAYOUT NOTES:
%
% The next \usepackage line changes the layout for CV style section
% headings as marginal notes. It also sets up the paper size as either
% letter or A4. By default, letter was used. If A4 paper is desired,
% comment out the letterpaper lines and uncomment the a4paper lines.
%
% As you can see, the margin widths and section title widths can be
% easily adjusted.
%
% ALSO: Notice that the includefoot option can be commented OUT in order
% to put the PAGE NUMBER *IN* the bottom margin. This will make the
% effective text area larger.
%
% IF YOU WISH TO REMOVE THE ``of LASTPAGE'' next to each page number,
% see the note about the +LP and -LP lines below. Comment out the +LP
% and uncomment the -LP.
%
% IF YOU WISH TO REMOVE PAGE NUMBERS, be sure that the includefoot line
% is uncommented and ALSO uncomment the \pagestyle{empty} a few lines
% below.
%

%% Use these lines for letter-sized paper
%\usepackage[paper=letterpaper,
%           %includefoot, % Uncomment to put page number above margin
%            marginparwidth=0.7in,     % Length of section titles
%            marginparsep=.05in,       % Space between titles and text
%            margin=0.5in,               % 1 inch margins
%            includemp]{geometry}

% Use these lines for A4-sized paper
\usepackage[paper=a4paper,
            %includefoot, % Uncomment to put page number above margin
            marginparwidth=24mm,    % Length of section titles
            marginparsep=1mm,       % Space between titles and text
            margin=15mm,              % 25mm margins
            includemp]{geometry}

%% More layout: Get rid of indenting throughout entire document
\setlength{\parindent}{0in}

%% This gives us fun enumeration environments. compactitem will be nice.
\usepackage{paralist}
\usepackage[shortlabels]{enumitem}
% \usepackage[misc]{ifsym}
%% Reference the last page in the page number
%
% NOTE: comment the +LP line and uncomment the -LP line to have page
%       numbers without the ``of ##'' last page reference)
%
% NOTE: uncomment the \pagestyle{empty} line to get rid of all page
%       numbers (make sure includefoot is commented out above)
%
\usepackage{fancyhdr,lastpage}
\pagestyle{fancy}
%\pagestyle{empty}      % Uncomment this to get rid of page numbers
\fancyhf{}\renewcommand{\headrulewidth}{0pt}
\fancyfootoffset{\marginparsep+\marginparwidth}
\newlength{\footpageshift}
\setlength{\footpageshift}
          {0.1\textwidth+0.1\marginparsep+0.1\marginparwidth-2in}
\lfoot{\hspace{\footpageshift}%
       \parbox{3.5in}{\, \hfill %
                    \arabic{page} of \protect\pageref*{LastPage} % +LP
%                    \arabic{page}                               % -LP
                    \hfill \,}}

% Finally, give us PDF bookmarks
\usepackage{color,hyperref}
\definecolor{darkblue}{rgb}{0.0,0.0,0.3}
\hypersetup{colorlinks,breaklinks,
            linkcolor=darkblue,urlcolor=darkblue,
            anchorcolor=darkblue,citecolor=darkblue}

%%%%%%%%%%%%%%%%%%%%%%%% End Document Setup %%%%%%%%%%%%%%%%%%%%%%%%%%%%


%%%%%%%%%%%%%%%%%%%%%%%%%%% Helper Commands %%%%%%%%%%%%%%%%%%%%%%%%%%%%

% The title (name) with a horizontal rule under it
%
% Usage: \makeheading{name}
%
% Place at top of document. It should be the first thing.
\newcommand{\makeheading}[1]%
        {\hspace*{-\marginparsep minus \marginparwidth}%
         \begin{minipage}[t]{\textwidth+\marginparwidth+\marginparsep}%
                {\large \bfseries #1}\\[-0.15\baselineskip]%
                 \rule{\columnwidth}{1pt}%
         \end{minipage}}

% The section headings
%
% Usage: \section{section name}
%
% Follow this section IMMEDIATELY with the first line of the section
% text. Do not put whitespace in between. That is, do this:
%
%       \section{My Information}
%       Here is my information.
%
% and NOT this:
%
%       \section{My Information}
%
%       Here is my information.
%
% Otherwise the top of the section header will not line up with the top
% of the section. Of course, using a single comment character (%) on
% empty lines allows for the function of the first example with the
% readability of the second example.
\renewcommand{\section}[2]%
        {\pagebreak[2]\vspace{1\baselineskip}%
         \phantomsection\addcontentsline{toc}{section}{#1}%
         \hspace{0in}%
         \marginpar{
         \raggedright \scshape #1}#2}

% An itemize-style list with lots of space between items
\newenvironment{outerlist}[1][\enskip\textbullet]%
        {\begin{itemize}[#1]}{\end{itemize}%
         \vspace{-0.6\baselineskip}}

% An environment IDENTICAL to outerlist that has better pre-list spacing
% when used as the first thing in a \section
\newenvironment{lonelist}[1][\enskip\textbullet]%
        {\vspace{-\baselineskip}\begin{list}{#1}{%
        \setlength{\partopsep}{0pt}%
        \setlength{\topsep}{0pt}}}
        {\end{list}\vspace{-.6\baselineskip}}

% An itemize-style list with little space between items
% \newenvironment{innerlist}[1][\enskip\textbullet]%
\newenvironment{innerlist}[1][\enskip$\circ$]%
        {\begin{compactitem}[#1]}{\end{compactitem}}

% An environment IDENTICAL to innerlist that has better pre-list spacing
% when used as the first thing in a \section
\newenvironment{loneinnerlist}[1][\enskip\textbullet]%
        {\vspace{-\baselineskip}\begin{compactitem}[#1]}
        {\end{compactitem}\vspace{-.6\baselineskip}}

% To add some paragraph space between lines.
% This also tells LaTeX to preferably break a page on one of these gaps
% if there is a needed pagebreak nearby.
\newcommand{\blankline}{\quad\pagebreak[2]}

% Uses hyperref to link DOI
\newcommand\doilink[1]{\href{http://dx.doi.org/#1}{#1}}
\newcommand\doi[1]{doi:\doilink{#1}}


%%%%%%%%%%%%%%%%%%%%%%%% End Helper Commands %%%%%%%%%%%%%%%%%%%%%%%%%%%

%%%%%%%%%%%%%%%%%%%%%%%%% Begin CV Document %%%%%%%%%%%%%%%%%%%%%%%%%%%%

%\hyphenpenalty = 9999
\def\vs{\vspace{-0.1in}}
\begin{document}
% \makeheading{Curriculum Vitae\\ [0.3cm] TIEP HUU VU\quad~~~~~~\quad\quad\quad\quad\quad\quad\quad\quad\quad\quad\quad\quad\quad\quad{\small Last update: December 17, 2015}}
\makeheading{Dana Ferranti}


\section{Contact Information}

\newlength{\rcollength}\setlength{\rcollength}{3 in}
\vs
\begin{tabular}[t]{@{}p{\textwidth-\rcollength}p{\rcollength}}
\texttt{Address} & {\large\Letter} \texttt{E-mail} \\
100 Institute Road & \href{mailto:dferranti@wpi.edu}{dferranti@wpi.edu}\\ 
Worcester Polytechnic Institute,      & \texttt{Website} \\
  Worcester, Massachusetts 01609    & \url{djferranti.github.io}
\end{tabular}
%% ==============================================================
\vspace{0.1in}
\section{Current position}
\begin{outerlist}
\item Assistant research professor in the Mathematical Sciences Department \hfill 2023 -  \\
Postdoctoral advisor: Dr. Sarah Olson
\end{outerlist}
\section{Research interests} % (fold)
\label{sec:research_backg}
\begin{outerlist}
\item Computational methods for viscous-dominated fluids described by the Stokes equations.
\item Biological applications of Stokes flow, including modeling of viscoelastic materials and biofilms.
\end{outerlist}
% section research_backg (end)
%% =========  ==============================
\vspace{0.1in}
\section{Education}
    \textbf{Tulane University}, New Orleans, LA \hfill 2017--2023
    \begin{outerlist}
    \item PhD, Mathematics.
    \item Thesis: \textit{Regularized Stokeslet surfaces and a coupled oscillator system in Stokes flow}  
    \item Advisor: Dr. Ricardo Cortez. 
    \end{outerlist}
    \vspace{0.1in}
\textbf{Clark University}, Worcester, MA. \hfill 2010--2014
    \begin{outerlist}
        \item BA, Mathematics and computer science.
    \end{outerlist}
    %% =========  ==============================
\section{Research Experience} % (fold)
\label{sec:research_exper}
\vspace{-0.25in}
\begin{outerlist}
\item {\bf Tulane University} \hfill 2017-2023 \\
  Center for Computational Science in Mathematics Department.
  \begin{innerlist}
    \item Extending the method of regularized stokeslets by using exact integration over triangulated surfaces. Minimal models of cilia interaction to investigating the potential effect of elastic coupling and inertia on synchronization.
  \end{innerlist}
    \item {\bf Massachusetts General Hospital} \hfill 2016--2017 \\
    Physics Research in Department of Radiation Oncology.
    \begin{innerlist}
    \item Using theoretical models to demonstrate the value of prior knowledge in determining causal relationships in complex networks, with applications to machine learning in medicine.
      \item Advisor: Dr. David Craft. 
    \end{innerlist}
\end{outerlist}

%% ================== block:  ==========================
\section{Teaching Experience}
\vspace{-.1in}
\textbf{As instructor}
\begin{outerlist}
\item Calculus IV \hfill WPI, Spring 2024 \\ \textit{Intro to multivariable calculus}
\item Probability \& Statistics I (Math 1110). \hfill Tulane University, Spring 2023 \\ \textit{Elementary probability theory and statistics} \\
  Recognized with \textbf{Outstanding Graduate Instructor award} given annually by Tulane University Math Department.
  \item Introduction to Applied Math (Math 2240). \hfill Tulane University, Fall 2021 \\ \textit{Ordinary differential equations for engineers/physicists}
  \end{outerlist}
  \vspace{0.5em}
  \textbf{As teaching assistant}
    \vspace{-0.1in}
  \begin{outerlist}
  \item Introduction to Applied Math (Math 2240). \hfill 2019, 2020, 2021 
  \item Linear algebra (Math 3090).  \hfill 2020
  \item Calculus I (Math 1210). \hfill 2017, 2019
  \item Calculus II (Math 1220). \hfill 2018, 2020
  \item Calculus III (Math 2210). \hfill 2018
  \end{outerlist}
  
\section{Service and Outreach}
\vspace{-0.25in}
\begin{outerlist}
\item President of AMS Graduate Student Chapter \hfill 2019-2021
\item Mathematics department  tea time organizer \hfill 2018-2022
\item Treasurer of AMS Graduate Student Chapter \hfill  2017-2019
\item Member of Inclusivity in Mathematics Task Force at Tulane (IMTF) \hfill 2020-2023
\end{outerlist}

\section{Talks}
\begin{outerlist}
\item \textit{Regularized Stokeslet Surfaces} \\
  Division of Fluid Dynamics (APS Meetings) in Washington D.C. (November 20, 2023)
\item \textit{Simulating bodies immersed in viscous flows: new developments
    in the Method of Regularized Stokeslets (MRS)} \\
  Worcester Polytechnic Institute Mathematics Colloquium (September 8, 2023)
\item \textit{Regularized Stokeslet Surfaces}
  Scientific Computing Around Louisiana (March 11, 2023)
\item \textit{Regularized Stokeslet Surfaces} \\
  Math for All in NOLA (February 25,2023)
\item \textit{An Extension to the Method of Regularized Stokeslets} \\
  Special session on Recent Developments in Numerical Methods for PDEs, Joint Math Meetings 2023 (January 4,2023)
  \item \textit{Computational Modeling of Bodies Immersed in Viscous Fluids} \\ Hunter College Applied Math Seminar (November 3,2022) 
\end{outerlist}
\section{Conferences}
\def\halfblankline{\vspace{0.1in}}
\begin{outerlist}
\item Division of Fluid Dynamics (APS Meetings) in Washington, D.C. (November 2023)
\item Joint Math Meetings in Boston, MA (January 2023)
\item SIAM Annual Meetings in Pittsburgh, PA (July 2022)
\item Blackwell-Tapia Conference at IMSI in Chicago, IL (Nov 2021)
\item Math for All in New Orleans (2020,2021,2023)
\item Scientific Computing Around Louisiana (2018, 2019, 2023)
% \href{mailto:rcortez@tulane.edu}{\textsf{rcortez@tulane.edu}}
\end{outerlist}

\section{Publications}
\begin{outerlist}
\item \textit{Regularized Stokeslet Surfaces} with Dr. Ricardo Cortez, 2023 \\
  accepted pending minor revisions to \textit{Journal of Computational Physics}
\item \textit{The value of prior knowledge in machine learning of complex network systems} \\
  with David Krane and Dr. David Craft (PI), \textit{Bioinformatics}, 2017
\end{outerlist}

% \section{References}
% \def\halfblankline{\vspace{0.1in}}
% \textbf{Dr. Ricardo Cortez}
% \begin{innerlist}
% \item[]
% Tulane University\\
% % \href{mailto:rcortez@tulane.edu}{\textsf{rcortez@tulane.edu}}
% \end{innerlist}


% \halfblankline
% \textbf{Dr. Michael Joyce}
% \begin{innerlist}
%  \item[]
%  Tulane University\\
%  \href{mailto:mjoyce3@tulane.edu}{\textsf{mjoyce3@tulane.edu}}
% \end{innerlist}

\end{document}

